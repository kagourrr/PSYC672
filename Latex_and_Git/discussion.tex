\section{Discussion}
The present study aimed to evaluate whether binaural beats could be used to induce neural resonance at specific frequencies and whether individual differences in ASD, as measured by the BAPQ, are associated with individual variations in neural resonance profiles. Overall, the results supported the hypotheses that binaural beats would induce a significant increase in oscillatory power above that observed while listening to a pure tone at specific frequencies and that there were significant relationships between peak resonance frequencies and ASD traits. These findings suggest that binaural beats can induce frequency-coupled neural resonance across participants at a wide range of frequencies. Furthermore, the findings suggest that neural resonance profiles are significantly associated with social and affective processes related to ASD.

