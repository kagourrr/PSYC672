\section{Method}
\subsection{Participants}
\par Younger adults (N = 57) were recruited from William \& Mary’s SONA (i.e., research participation) system. The sample was 72\% female, and the average age was 18.8 years (SD = .94). Participants were offered class credit for the approximately 75-minute-long procedure. Participants were excluded if they self-reported any auditory impairments, a history of stroke, or neuropsychiatric or neurodegenerative disorder. Five participant data files were also excluded from the analysis because of excessive noise in the EEG recordings (i.e., movement affected brain data). The experiment was conducted with the informed consent of each participant following a College of William \& Mary-approved IRB protocol (PHSC-2024-08-28-17197). 
\subsection{Procedure \& Stimuli }
Binaural beats were used to drive neural entrainment at frequencies between 1 Hz and 50 Hz, increasing in 1 Hz increments, centered around a frequency of 440 Hz. All auditory stimuli were generated using MATLAB (v2024a) and presented with ABR Tubephone insert earphones. As a control condition, participants were initially exposed to a 440 Hz pure-tone auditory stimulus. Following the pure tone stimulus, participants were presented with binaural beat stimuli at integer frequencies from 1 to 50 Hz, with the presentation of the frequencies occurring in random order. 