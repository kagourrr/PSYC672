\section{Introduction}
\par Auditory stimuli across frequency bands should establish a rhythm for oscillatory signals to entrain, providing opportunities to measure neural resonance. This phenomenon of neural resonance means that neural networks are in a steady state of excitability when stimuli are presented, and certain stimulation frequencies should induce ‘resonance peaks’ \citep{calderone2014entrainment}. Understanding how individual differences in neural resonance entrainment are related to ASD is critical because research suggests that neural resonance can be affected by factors like dendritic branching and network complexity, which can increase with age and cause a decrease in oscillatory coherence and synchronization, affecting ASD populations disproportionately \citep{laudanski2014spatially, martorell2019multi}. Exploring oscillatory power, neural resonance, and individual variability in neurodivergent and neurotypical individuals is critical for understanding how ASD affects social, affective, and cognitive functions. 
\par In the current study, binaural beats were used to induce neural entrainment within the frequency range of 1 to 50 Hz. If binaural beats are ineffective at inducing entrainment, one would expect no differences in oscillatory power at the binaural beat frequency between EEG recorded during the binaural beat stimulation and a pure tone condition (beat frequency = 0). Likewise, oscillatory power, reflecting neural resonance, would be expected to increase during binaural beat stimulation in participants with higher levels of ASD traits such as aloofness, comparable to those with lower levels of these traits, suggesting no significant difference in neural resonance profiles based on ASD trait severity. The first alternative hypothesis is that binaural beat stimulation does induce a significant increase in oscillatory power compared to the pure tone. This hypothesis means that there would be peak frequencies for different binaural beat stimulation frequencies (1-50 Hz) that are significantly higher in amplitude compared to pure tone stimulation frequencies (100 Hz). The second alternative hypothesis is that there are significant differences between the resonance profiles of participants high on the ASD scale versus low on the ASD scale. This hypothesis means that we would expect significantly higher peak frequencies in different canonical frequency bands (i.e., Alpha, Beta, Theta, Gamma, and Delta) for participants scoring higher on the ASD scale. 
